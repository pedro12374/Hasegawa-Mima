
\documentclass[12pt, a4paper]{article}
\usepackage{graphicx}               %inclusão de figuras
\usepackage{amsmath, amssymb}       %inclusão de símbolos matemáticos 
\usepackage{lmodern}                %habilita alguns caracteres latinos extras
\usepackage[T1]{fontenc}            %permite acentuação
\usepackage[utf8]{inputenc}         %idem anterior - no windows usar latin1
\usepackage[brazil,brazilian]{babel}%habilita hifenizações
\usepackage[margin= 2.5cm]{geometry}%define o tamanho das margens
\usepackage[normalem]{ulem}         %permite sublinhar uma palavra
\usepackage{setspace}               %permite usar espaçamento \onehalfspacing \doublespacing
\usepackage{indentfirst}            %indenta o primeiro parágrafo
\usepackage{color, colortbl}        %permite textos e tabelas coloridas
\usepackage{booktabs, array}        %comandos adicionais para melhorar a qualidade das tabelas
\usepackage[hidelinks]{hyperref}               %gera hiperlinks no pdf 


\usepackage[fixlanguage]{babelbib}
\selectbiblanguage{brazil}
\usepackage{caption}

\usepackage{physics}



\author{Pedro Haerter}
\title{Wave Coupling}
\date{\today}
\begin{document}

%capa
\maketitle
\onehalfspacing 

\section{Hasegawa-Mima equation}

\begin{equation}
	\pdv{t}(\nabla^2\phi-\phi)-\qty[(\nabla\phi\cp\vb{\hat{z}})\cdot\nabla]\qty[\nabla^2\phi-\ln\qty(\frac{n_0}{\omega_{ci}})]=0
	\label{eq:hasegawa-mima}
\end{equation}

The Hasegawa-Mima equation \ref{eq:hasegawa-mima}  describes the drift-wave propagation of waves in plasma and the emergence of cross-field transport, the equation is a starting point in the study of instability in plasma drift waves and how it can decay in two or more coupled waves, that emerge when the initial waves reach a limit value

The equation is based on the propagation of electrostatic waves with frequency $\omega\ll\omega_{ci}$, where $\omega_{ci}=eB/m_i$ is the ion cyclotron frequency. The wave is propagated in a magnetized and in-homogeneous plasma medium.  The wave is then treated as a drift wave.

One of the solutions to this equation is using to consider the potential as an infinite sum of waves in the form of 

\begin{equation}
	\phi(\vb{r},t)=\frac{1}{2} \sum_{\vb{k}=1}^{\infty}\qty[\phi_{\vb{k}}(t)e^{i\vb{k}\vdot\vb{r}}+\phi^*_{\vb{k}}(t)e^{-i\vb{k}\vdot\vb{r}}]
\end{equation}
the equation that describes $\phi_{\vb{k}}$ is 

\begin{equation}
	\dv{\phi_{\vb{k}}}{t}=-i\omega_{\vb{k}}\phi_{\vb{k}}+\sum_{\vb{k}=\vb{k}_2+\vb{k}_3}\Lambda_{\vb{k}_2,\vb{k}_3} \phi_{\vb{k}_2}^*\phi_{\vb{k}_3}^* +\gamma_{\vb{k}}\phi_{\vb{k}}
	\label{eq:sol_hwm}
\end{equation}
where $\Lambda$ is a constant factor that depend on $\vb{k},\vb{k}_2,\vb{k}_3$, and $\gamma$ is a dissipative term.

\section{$\vb{E}\cp\vb{B}$ drift}

Considering a test particle embedded in a nonuniform plasma and using the slab approximation. The movement of the particle is given by

\begin{equation}
	\vb{v}_E=c\frac{\vb{E}\cp\vb{B}}{B^2}
\end{equation}

The nonuniformity is given by electrostatic potential in the plasma,  $\vb{E}=\grad\phi(\vb{r},t)$. Considering a uniform magnetic field in the $\hat{z}$ direction and a perpendicular electric field the velocity became

\begin{equation}
 	\vb{v}_E=-\frac{1}{B_0}\grad\phi(x,y,t)\cp\hat{z}
 \end{equation} 
 then the $x$ and $y$ velocities are

 \begin{equation}
  	v_x = \dv{x}{t}=-\frac{1}{B_0}\pdv{y}\phi(x,y,t)\qquad \text{and}\qquad v_y = \dv{y}{t}=\frac{1}{B_0}\pdv{x}\phi(x,y,t)
  \end{equation} 

  Comparing these equations with the Hamilton equation, it's notable that

  \begin{equation}
  	H(x,y,t)=\frac{\phi(x,y,t)}{B_0}
  	\label{eq:Hamilton}
  \end{equation}




With this set of equations, we can plug the solution of the Hasegawa-Mima equation into the movement equation to particles to study how the particles inside a tokamak are affected by drift waves described by the equation \ref{eq:hasegawa-mima}. 

\section{Three Waves}

A more general system is using three wave modes produced by the equation \ref{eq:sol_hwm}. Considering the notation $\phi_{\vb{k}_1}$ as $\phi_1$, the ODE to $\phi$ are

\begin{align}
	\dv{\phi_{1}}{t}=-i\omega_{1}\phi_{1}+\Lambda^1_{2,3}\phi_2^*\phi_3^*+\gamma_1\phi_1,\\
	\dv{\phi_{2}}{t}=-i\omega_{2}\phi_{2}+\Lambda^2_{3,1}\phi_3^*\phi_1^*+\gamma_2\phi_2,\\
	\dv{\phi_{3}}{t}=-i\omega_{3}\phi_{3}+\Lambda^3_{1,2}\phi_1^*\phi_2^*+\gamma_3\phi_3
\end{align}
Then the total potential is

\begin{equation}
	\phi(\vb{r},t)_{HM}=\frac{1}{2}\qty[\phi_1e^{i\vb{k}_1\vdot\vb{r}}+\phi_2e^{i\vb{k}_2\vdot\vb{r}}+\phi_3e^{i\vb{k}_3\vdot\vb{r}}+\phi_1^*e^{-i\vb{k}_1\vdot\vb{r}}+\phi_2^*e^{-i\vb{k}_2\vdot\vb{r}}+\phi_3^*e^{-i\vb{k}_3\vdot\vb{r}}]
\end{equation}
simplifying
\begin{equation}
	\phi(\vb{r},t)_{HM}=\sum_{\vb{k}=1}^{3}\qty[\Re{\phi_{\vb{k}}}\cos(\vb{k}\vdot\vb{r})-\Im{\phi_{\vb{k}}}\sin(\vb{k}\vdot\vb{r}) ]
\end{equation}
where the $HM$ represent the Hasegawa-Mima potencial wave. 

Using the equation \ref{eq:Hamilton} to describe the velocity of the guiding centers with the potential $\phi_{HM}$ we had the result

\begin{figure}[!h]
	\centering
	\includegraphics[width=\textwidth]{fig/Difusion.png}
	\caption{Eletrostatic Wave potential, Phase Space, movement in x direction and movement in y direction using just the turbulent potential}
	\label{fig:difu}
\end{figure}

where the particles in the radial direction represented by $x$ diverge to infinity due the turbulence nature of the process. To represent the confinement seen in real Tokamaks we can introduce a attraction term in the, represented by a confinement potential $\phi_c=-kx$, where it will play the role of a restaurative force in classical dynamics, then our equations of motion becomes

\begin{equation}
	v_x = \dv{x}{t}=-\frac{1}{B_0}\pdv{y}\phi_{HM}(x,y,t)\qquad \text{and}\qquad v_y = \dv{y}{t}=\frac{1}{B_0}\pdv{x}\phi_{HM}(x,y,t)-k
\end{equation}
applying this extra potential our diffusive escape is eliminated and the particles are confined, as we can see in figure \ref{fig:conf}

\begin{figure}[!h]
	\centering
	\includegraphics[width=\textwidth]{fig/conf.png}
	\caption{Eletrostatic Wave potential, Phase Space, movement in x direction and movement in y direction using the linear potential of confinement}
	\label{fig:conf}
\end{figure}

If we use an quadratic potential, the equation of movement becomes
\begin{equation}
	v_x = \dv{x}{t}=-\frac{1}{B_0}\pdv{y}\phi_{HM}(x,y,t)\qquad \text{and}\qquad v_y = \dv{y}{t}=\frac{1}{B_0}\pdv{x}\phi_{HM}(x,y,t)-kx
\end{equation}
and the confinement also occurs and we can see some better patterns, as in figure \ref{fig:conf2}

\begin{figure}[!h]
	\centering
	\includegraphics[width=\textwidth]{fig/quad.png}
	\caption{Eletrostatic Wave potential, Phase Space, movement in x direction and movement in y direction using the quadratic potential of confinement}
	\label{fig:conf2}
\end{figure}



\end{document}